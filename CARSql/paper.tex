\documentclass[CJK]{ctexart}
\CTEXsetup[format={\large\bfseries}]{section}
\usepackage[top=2.5cm, bottom=2.5cm, left=2.5cm, right=2.5cm]{geometry}
\usepackage{listings}
\usepackage{fontspec}
\usepackage{xcolor}
\usepackage{graphicx}
\usepackage{indentfirst}
\usepackage{amssymb}
\usepackage{multicol}
\usepackage[superscript]{cite}
\usepackage{xeCJK}
\setmainfont{Times New Roman}
\setmonofont{Consolas}
\setCJKmainfont{simsun.ttc}

\pagestyle{plain}
\newenvironment{figurehere}
{\def\@captype{figure}}
{}
\newenvironment{tablehere}
{\def\@captype{table}}
{}
\lstset{
 columns=fixed,
 frame=none,
 backgroundcolor=\color[RGB]{245,245,244},
 keywordstyle=\color[RGB]{40,40,255},
 numberstyle=\footnotesize\color{darkgray},
 commentstyle=\it\color[RGB]{0,96,96},
 stringstyle=\color[RGB]{128,0,0},
 showstringspaces=false,
 language=java,
 tabsize=4,
 basicstyle = \footnotesize\ttfamily,
 breaklines=true,
}

\makeatletter
\newcommand\figurecaption{\def\@captype{figure}\caption}
\newcommand\tablecaption{\def\@captype{table}\caption}
\makeatother
\begin{document}
\begin{center}
{\LARGE \textbf {大数据分析技术论文}}
\end{center}

\begin{center}
署名写两行吧,把姓名学号都写上
\newline
\end{center}
\setlength{\columnsep}{1cm}
\begin{multicols}{2}
\begin{center}
{\large \textbf {摘要}}
\end{center}
    这里写摘要。整完了一起写摘要。
\section{项目背景}
{\normalsize {
    该项目基于某地区一个月内的路口车辆监控数据。每条数据表示为[car, crossroad, timestamp],分别表示车辆标号,路口编号和时间戳。
    在这31的数据中,共有2.7亿条记录,文件大小约为5.7GB。不同的车辆约有2300万,路口数量约为1000。

    项目的任务是寻找出所有的伴随车,即总是同时出现在的同一路口的车辆。
    对数量如此巨大的数据集进行挖掘,显然我们不能再着眼于传统的解决方法,而需要利用特定的技术来进行数据处理。

}}
\section{相关工作}
\subsection{Spark}
\subsubsection{产生背景}
    作为分布式计算的经典解决方案,Mapreduce由于对HDFS要进行频繁的操作(如计算结果持久化,数据备份,shuffle等),导致磁盘IO成为了系统性能的瓶颈,因此Mapreduce对于迭代式,交互式以及流式数据的处理就显得力不从心。
\subsubsection{Spark简介}
Spark是一个开源的分布式计算框架,对Java,Scala提供了高层的API,并且有一个经优化的支持通用执行图计算的引擎。与传统MapReduce分布式计算模型相比,spark速度更快的原因是:

1.MapReduce基于的Hadoop需要频繁把结果落盘到hdfs上,网络与磁盘IO开销非常大,但是spark可以把必要的运算结果缓存到内存中,从而减少了磁盘IO开销。

2.在复杂的运算中,数据的运算需要经历多个步骤,而spark可以把每个操作抽象为一个operator,然后这些操作串成一个有向无环图,进而优化计算路径。

3.Spark引入了弹性分布式数据集RDD,RDD可以理解为一个分布式对象集合,每个RDD可以分为多个分区,每个分区就是一个数据集片段,一个RDD的不同分区可以保存到集群中的不同节点上,从而可以实现不同节点上的并行计算。

RDD的特点:

1.只读:不能修改,只能通过转换操作生成新的RDD。

2.分布式:可以分布在多台机器上并行处理。

3.弹性:计算过程中内存不够的时候会与磁盘进行数据交换。

4.基于内存:可以全部或者部分缓存在内存中,在多次计算间重用。
\subsubsection{Spark SQL}
Spark SQL主要用于结构化的数据处理,其本身也可以作为一个分布式的查询引擎,即可以同时对存储于多个节点上的数据做查询。

Spark SQL之所以速度如此之快,是因为它是一种纯内存计算模型,基本上消灭了数据磁盘寻道和IO的时间开销,而磁盘的overhead恰恰是Mapreduce和Hive的性能瓶颈所在。
\subsubsection{Spark SQL特点}

1.SQL可以和Spark编程模型完美融合,结构化的数据可以在Spark程序内使用SQL或者DataFrame API同时进行处理,一方面利用了SQL的简单语法便于对数据范式进行组织,另一方面又有利于使用函数式编程模型来进行复杂的数据分析。

2.大大优化了SQL的Interpreter和Optimizer,整合了 cost-based optimizer, code generation and columnar storage 使得在进行并行计算的同时可以保证查询语句的高性能。

\section{方法A}
如果两辆车出现在同一路口的时间相近,则可以认为这两辆车伴随一次。对于总是伴随的多辆车,认定为一组伴随车队。

对于当前任务,我们期望获取记录中的多个伴随车队。而要获取所有的伴随车对,就需要对整个原始数据进行分析。

该算法将首先对原始数据进行分割成31份文件,每个文件对应一天的数据。接着通过一些阈值对数据进行过滤,再对过滤后的记录做数据挖掘处理。

这里用到的数据挖掘方法的核心思想是通过把数据读取成Spark Sql的一张表,然后使用对应的sql语句查询出所有伴随车的二元组以及其伴随次数,保存到中间文件。再对中间文件进行进一步处理,输出相关结果。
\subsection{数据分割和过滤}
原始文件共有2.7亿条数据,受限于设备的内存性能,对这样的数据进行直接排序比较困难。考虑到这样的情况,我们选择只对所有记录进行筛选,将数据分成31份。

这一步准备工作,由单机完成。通过python的相关数据处理库pandas。我们的设备读入原始文件,并对时间戳做筛选,将对应日期的记录输出到对应的文件。

这样获取的31份文件内部未排序,平均每个文件中包含800万条记录。在Linux中,利用命令行对每个文件按照时间、路口作为关键词进行排序。

经过我们对数据的初步感知,结果中的伴随车伴随次数应该至少为千级,且对800万条数据做处理依然是一个比较困难的问题,我们希望进一步在可控范围内控制文件中记录的数量。我们首先对每个文件按照路口进行归并,并统计每个路口当日
的车辆记录数量。事实上,在现实情况下,我们去掉某一个路口的数据,是不会对最后的伴随车判断结果造成影响的。

由于记录数越多的路口,计算起来越困难,所以我们会将每天数量最多的6个路口的数据去除。

除此之外,对于剩下的数据,我们将统计每辆车出现的次数。因为结果中伴随车的出现次数必然大于伴随次数,则我们可以认为出现次数比较少的车,是不大可能成为伴随车的。所以,对于当日出现次数小于10的车辆,我们也会删去对应的记录。

经过这样的处理之后,将31个文件剩下的记录合并,得到大约280万条记录。最终的结果就由这280w条数据产生。

\subsection{数据读取}
原始文件中的每一条数据,都有三个项[car, crossroad, timestamp]用逗号隔开,且没有经过任何排序。
我们通过java将每一条数据读取为carinfo的数据结构,以便下一步的处理。
\begin{lstlisting}
public static class carInfo implements Serializable {
    private int carID;
    private int roadID;
    private Long timestamp;
}
\end{lstlisting}
读取hdfs上的源文件,转化为Spark的RDD数据类型,然后调用map,重载一个转换函数,把csv的每一行转化为一个CARINFO对象。然后把RDD转化为Spark Sql的一个Dataset,然后创建一个Spark Sql的"car"表。
\begin{lstlisting}
JavaRDD<carInfo> carRDD = spark.read()
    .textFile("hdfs://10.141.212.225:9000/"+file)
    .javaRDD()
    .map(new Function<String, carInfo>() {
        @Override
        public carInfo call(String lines) throws Exception {
            String[] line = lines.split(",");
            int lenTimestamp = line[2].length();
            carInfo CARINFO = new carInfo();
            CARINFO.setCarID(Integer.parseInt(line[0]));
            CARINFO.setRoadID(Integer.parseInt(line[1]));
            CARINFO.setTimestamp(Long.parseLong(line[2].substring(0,lenTimestamp-1)));
            return CARINFO;
        }
    });
Dataset<Row> carDF = spark.createDataFrame(carRDD, carInfo.class);
carDF.createOrReplaceTempView("car");
\end{lstlisting}
\subsection{获取伴随车组}
将数据读入后,我们需要获取所有的伴随车。如果前后两辆车在同一路口出现并且出现时间相差在90秒内,就判定为一次伴随。
这一步,我们将对读入的所有carinfo进行处理,得到所有的伴随车对。
\begin{lstlisting}
Dataset<Row> carId1DF = spark.sql(
    "SELECT car1.carID as carID1, car2.carID as carID2  " +
    "from car car1, car car2 " +
    "WHERE car1.roadID = car2.roadID " +
    "AND ABS(car1.timestamp-car2.timestamp) < 90 " +
    "AND car1.carID != car2.carID");
carId1DF.createOrReplaceTempView("accompany");
\end{lstlisting}


\subsection{数据去重并统计伴随次数}
显然,上一步得到的数据中必定存在大量重复。所以我们对每一个伴随车对,统计它们的伴随次数,以判断它们是否较为频繁出现伴随现象。
结果中只保留伴随次数超过10000的伴随车对,作为我们的结果并保存到result-all.txt中。
\begin{lstlisting}
Dataset<Row> carId2DF = spark.sql(
    "SELECT carID1, carID2 " +
    "from accompany a " +
    "WHERE (a.carID1, a.carID2) in " +
    "(select carID1, carID2 from accompany " +
    "group by carID1, carID2 " +
    "having count(*) > 10000)"
    );
\end{lstlisting}


\subsection{项目实现}
\subsubsection{实验环境}
    数据初步分割和过滤部分基于单机进行,运行环境为。

    我们一共使用了5台server,ip为10.141.212.223-227。基于这五台设备,配置hadoop和spark,其中一台为master,其他为slave。
    hadoop版本为,spark版本为。
\subsubsection{实验过程}
    在3台Spark节点上运行该程序,总耗时约5分钟对280多万条数据进行处理,其中包括3个sql语句的执行,两个中间表的生成以及最终结果保存到文件的过程。
    相关代码见github。
\subsubsection{实验结果}
    实验得到的伴随车对数据共有条,其中伴随次数最高为62万余次。

\section{性能分析}
    整个生成伴随车对的核心部分,共耗时10分钟左右。相较于要是数据而言,其性能已经非常卓越。如此卓越的性能,主要得益于Spark和sql的结合实现过程。

\subsection{spark SQL的执行过程}
spark SQL的执行过程主要分为以下三步

1.首先经过SQL Parser对SQL进行词法分析,语法分析,语义分析等生成语法树

2.Catalyst优化器对语法树进行进一步优化处理

3.最后转化为RDD的一个有向无环图交给Spark执行

其中,sparkSQL高性能的核心是Catalyst Optimizer,可以把其功能类比为spark的DAG scheduler。

DAG scheduler 负责把RDD构建成一个有向无环图,并且给出一套执行RDD的执行计划。而Catalyst Optimizer则用来把构造Dataset的变换操作转化为sql执行的物理计划,本质上就是优化从Dataset转化为RDD的执行计划的过程。

Catalyst Optimizer进行的过程可以划分为:

1、Analysis: 主要把parser转化后的Unresolved Logical Plan转化为Analyzed Logical Plan

2、Logical Optimization: 利用一些优化规则把Logical Plan转化为Optimized Logical Plan

3、Physical Planning:逻辑计划不能直接被Spark执行,需要利用代价模型(cost model)选择性能最佳的physical plan

4、代码生成器把SQL查询生成Java的字节码

\subsection{Catalyst Optimizer}
1.Analysis

Analysis处理的基本数据结构有两种:SQL parser返回的抽象语法树或者是程序中API创建的DataFrame或者Dataset.这里的分析主要是要解决未解析的属性引用或者关系:

`select carId from car;`

carId这个column的类型是什么以及其是否有效,在我们查找表之前都是不知道的,也就是说这个属性是未解析的。首先构建一个包含未绑定属性和数据类型的“Unresolved Logical Plan”树,然后使用Catalyst规则和跟踪所有数据源中的表的Catalog对象来解析这些属性。

2.Logical Optimization

上一个步骤得到的Analyzed Logical Plan可以直接转化为Physical Plan然后在Spark中执行,但是这样得到的Physical Plan不一定是最优的。因此要进行逻辑优化以得到更优的逻辑算子语法树,这是完全基于预先设定的启发式规则的。

常见规则包括:

列裁剪(过滤掉查询不需要使用的列);

谓词下移(把where过滤操作尽可能下沉到数据源端,尽早进行数据的过滤);

常量折叠(预先计算好常量);

Bool表达式简化等。

3.Physical Planning

逻辑计划是不能直接在Spark执行的,所以必须要翻译成物理计划。一个逻辑计划经过一系列处理后会得到许多物理计划,在Spark sql中是通过SparkPlan实现的。多个物理计划再经过代价模型(Cost Model)得到选择后的物理计划(Selected Physical Plan)。

Cost Model核心思想是计算每个物理计划的代价,然后选择代价最少的作为候选计划。

4.Code Generation

最后把物理计划转化为Java的字节码,这里Catalyst利用了Scala的语言特性:Quasiquotes(预引用)

Quasiquotes允许在Scala语言中编程构造抽象语法树(AST),然后可以在运行时将其提供给Scala编译器以生成字节码。我们使用Catalyst将表示SQL中表达式的树转换为AST,以便Scala代码评估该表达式,然后编译并运行生成的代码。

换句话说,当我们需要访问某些对象的字段的时候,我们可以直接通过代码生成直接访问这些字段,就不需要把对象复制到Spark并存储为Row,即预引用大大加速了代码生成以及评估这一过程。
\end{multicols}
\end{document}
